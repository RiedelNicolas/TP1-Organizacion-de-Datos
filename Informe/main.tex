\documentclass[titlepage,a4paper]{article}

\usepackage{a4wide}
\usepackage[colorlinks=true,linkcolor=black,urlcolor=blue,bookmarksopen=true]{hyperref}
\usepackage{bookmark}
\usepackage{fancyhdr}
\usepackage[spanish]{babel}
\usepackage[utf8]{inputenc}
\usepackage[T1]{fontenc}
\usepackage{graphicx}
\usepackage{float}
\usepackage{tabularx}
\usepackage{tabto}
\usepackage{pgfgantt}
\usepackage{xcolor}
\usepackage{wrapfig}
\usepackage[utf8]{inputenc}
\usepackage{graphicx}
\usepackage{subcaption}



\pagestyle{fancy} % Encabezado y pie de página
\fancyhf{}
\fancyhead[L]{El cuarteto imperial - TP1}
\fancyhead[R]{75.06 Organización de Datos - FIUBA}
\renewcommand{\headrulewidth}{0.4pt}
\fancyfoot[C]{\thepage}
%\renewcommand{\footrulewidth}{0.4pt}

\begin{document}
\begin{titlepage} % Carátula
    
	\hfill\includegraphics[width=6cm]{logofiuba.jpg}
    \centering
    \vskip1cm
    \Huge \textbf{UBA - Facultad de Ingeniería}
    \vskip0.25cm
    \LARGE{Departamento de Computación}    
    \vskip0.25cm
    \LARGE{Organización de Datos (75.06)}
    \vskip1.2cm
    \vskip0.3cm
    \Huge \textbf{Trabajo Práctico 1} 
    \vskip0.5cm
    \LARGE{1er cuatrimestre - 2020}
    \vskip1.5cm
    \large
  	\begin{center}
    \begin{tabular}{||{7cm}||{2cm}||{6cm}||}
     \hline
     \multicolumn{3}{||c||}{GRUPO: \textbf{El Cuarteto Imperial}} \\ [0.5ex]
     \hline
     \hline
     \centering{\textbf{Alumno}} & \textbf{Padrón} & \textbf{Mail}\\ \hline
          LARREA BUENDÍA, Hugo Marcelo & xxxx & xxxx@fi.uba.ar\\ \hline
          MARTINEZ SASTRE, Gonzalo Gabriel & 102321 & \normalsize gonzalomartinezsastre@gmail.com \\ \hline
          RIEDEL, Nicolás Agustín & 102130 & nriedel@gmail.com\\ \hline
          ZBOGAR, Ezequiel & 102216 & ezbogar@fi.uba.ar\\ \hline
    \end{tabular}
    \end{center}
     
    \end{titlepage}
    
\tableofcontents
\newpage
\setlength{\parskip}{2mm}
\section{Resumen}\label{sec:resumen}
Nuestro Trabajo Práctico se basa en un circuito electrónico controlado por una computadora central programable llamada Arduino, que emula el juego SIMON.  

Teníamos como objetivo utilizar los conocimientos adquiridos en la materia y puestos en práctica con las fichas de laboratorio para llevar a cabo el proyecto.    

Nuestra intención era lograr mediante la realización del trabajo familiarizarnos con el funcionamiento y desarrollo de un sistema que use un microcontrolador programable por su amplia versatilidad y aplicabilidad. 



\newpage

\section{Introducción}\label{sec:intro}
    El objetivo de este informe es exponer el proyecto del juego SIMON realizado por el grupo 2 de la materia Laboratorio(66.02). Se detallarán los procesos llevados a cabo para su diseño y construcción, además de las mediciones realizadas que se encuentran relacionadas con los contenidos teóricos relevantes a este curso.
    
    \begin{figure}[H]
    \centering
    \includegraphics[width=0.4\textwidth]{SIMON.jpg}
    \caption{\label{fig:class01}El juego SIMON.}
    \end{figure}

    \subsection{Consideraciones y propuestas analizadas}
        Inicialmente se tenía planeado, además de hacer el SIMON, construir también una fuente de alimentación. Finalmente se decidió con ayuda del JTP y ayudantes del Labi que era una complicación innecesaria y simplemente se siguió adelante con el proyecto del juego sin la fuente. Para alimentarlo, se utiliza el puerto usb de una computadora, o en su defecto, una batería de 9V.
        Por ultimo, se consideró, utilizando el mismo hardware, agregarle otros juegos con un selector al inicio pero al observar que no agregaría nada de interés relevante a los temas estudiados se optó por no hacerlo.
    
    \subsection{Versión final del proyecto}
    Finalmente, una vez determinado cómo realizar el proyecto, se diagramó aproximadamente cómo estarían ubicados los componentes y las líneas de soldadura. Se diseñaron programas de prueba con el software de Arduino para verificar el correcto funcionamiento de los componentes, y cuando todo estaba funcionando correctamente, se procedió a soldar todo a la placa.
    
        
    

\end{document}
